
\documentstyle[11pt]{article}
\newcounter{enumval}

\textwidth     6.5in
\textheight      9in
\topmargin    -0.5in
\oddsidemargin   0in
\evensidemargin  0in
\parskip         5pt plus 3pt minus 1pt
\parindent       0pt

\newcommand{\luna}{{\sf LUNA}}
\newcommand{\moon}{{\sf MOON}}
\newcommand{\sun }{{\sf SUN }}

\newcommand{\xf}[1]{Figure~\ref{#1}}
\newcommand{\xs}[1]{Section~\ref{#1}}
\newcommand{\id}[1]{\mbox{\it #1\/}}
\newcommand{\kw}[1]{\mbox{\sf #1}}
\newcommand{\qt}[1]{``{\tt #1}''}
\newcommand{\cmt}{{\rm \%}}
\newcommand{\usc}{\underline{\hskip 1em}}
\newcommand{\bt}{$\bullet$}
\newcommand{\lr}{\mbox{$\longrightarrow$}}
\newcommand{\code}[1]{\langle#1\rangle}

%\newcommand{\cmem}[1]{{\cal M}(#1)}
%\newcommand{\creg}[1]{{\cal R}(#1)}

\setcounter{secnumdepth}{3}

\newenvironment{prog}{\par\def\-{\qquad}\vspace{\parskip}%
\parskip0pt\leftskip2em\obeylines}{}

\newcommand{\bmem}[1]{{\cal M}_8[#1]}
\newcommand{\wmem}[1]{{\cal M}_{32}[#1]}
\newcommand{\breg}[2]{{\cal R}_{#1}(#2)}
\newcommand{\wreg}[1]{{\cal R}(#1)}
\newcommand{\ic}{\mbox{\it PC\/}}
\newcommand{\ass}[1]{\stackrel{#1}{\longleftarrow}}
\newcommand{\nont}[1]{\mbox{\sl #1\/}}
\newcommand{\ang}[1]{\langle\mbox{\it #1\/}\rangle}

\newenvironment{nolab}%
{\begin{list}{}{\topsep 0pt plus 3pt \labelsep 0.5em%
\leftmargin 2em \parsep0.5ex \itemsep 0ex}}%
{\end{list}}

\author{Peter Grogono \\ \\
Department of Computer Science\\
Concordia University}
\title{The \moon\ Processor}
\date{1995}
\pagestyle{headings}
\begin{document}
\maketitle
% Heading: change to your requirements

%\begin{center}
%{\Large COMP 442/642 Compiler Design}\\[1pc]
%{\large Winter 1995}\\[3pc]
%\end{center}

\section{The \moon\ Processor}

The \moon\ is an imaginary processor based on recent RISC
architectures.\footnote{A \moon\ is similar to a \sun, but not as
bright.} The architecture is similar to, but simpler than, the DLX
architecture described by John Hennessy and David Patterson in their
textbook.\footnote{{\it Computer Architecture: a Quantitative
Approach\/}, John Hennessy and David Patterson, Morgan Kaufmann, 1990.}
This document describes the architecture, instruction set, and assembly
language of the \moon\ processor.

\subsection{Architecture}

The \moon\ is a RISC (Reduced Instruction Set Computer). The number of
different instructions is small, and individual instructions are simple.
All instructions occupy one word and require one memory cycle to execute
(additional time may be required for data access).

\subsubsection{Processor}

The processor has a few instructions that access memory and many
instructions that perform operations on the contents of registers. Since
register access is much faster than memory access, it is important to
make good use of registers to use the \moon\ efficiently.

There are sixteen registers, $R0,R1,\ldots,R15$. $R0$ always contains
zero. There is a 32-bit {\it program counter\/} that contains the
address of the next instruction to be executed.

\subsubsection{Memory}

A memory address is a value in the range $0,1,\ldots,2^{31}$. The amount
of memory actually available is typically less than this.

Each address identifies one 8-bit byte. The addresses $0,4,\ldots,4N$
are {\it word addresses\/}. The processor can load and store bytes and
words.

\subsection{Terminology and Notation}

A {\bf word} has 32 bits. The bits are numbered from 0 (the most
significant) to 31 (the least significant).

An {\bf integer} is a 32-bit quantity that can be stored in a word.
An integer value $N$ satisfies the inequality $2^{-31} \le N < 2^{31}$.
Bit~0 is the {\bf sign bit}. Integers are stored in two's-complement
form.

An {\bf address} has 32 bits. Address calculations may involve signed
numbers, but the result is interpreted as an unsigned, 32-bit quantity.

A {\bf byte} has 8 bits. The bits are numbered from 0 (the most
significant) to 7 (the least significant). Up to four bytes may be stored
in a word.

The name of the memory is ${\cal M}$. The expression $\bmem{K}$ denotes
the byte stored at address $K$. The expression $\wmem{K}$ denotes the
word stored at addresses $K,K+1,K+2$, and $K+3$.

An address is {\bf legal} if the addressed byte exists. Legal addresses
form a contiguous sequence $0,1,\ldots,N$, where $N$ depends on the
processor or simulator.

An address is {\bf aligned} if it is a multiple of 4. The aligned
addresses are therefore $0,4,8,\ldots,4N$.

The names $R0,R1,\ldots,R15$ denote {\bf registers}. Each register can
store a 32-bit word. We write $\wreg{i}$ to denote the contents of
register $Ri$ and $\breg{a..b}{i}$ to denote the contents of bits
$a,a+1,\ldots,b$ of register $Ri$. At all times, $\wreg{0}=0$.

The name \ic\ denotes the {\bf program counter}. The program counter
stores the 32-bit address of the current instruction.

The symbol $\longleftarrow$ stands for data transfer, or assignment. A
numeric superscript indicates the number of bits transferred. For
example, $ \breg{24..31}{3} \ass{8} \bmem{1000} $ means that 8 bits (one
byte) is transferred from memory location $1000$ to the least
significant byte of $R3$.

\subsection{Instruction Set}

\subsubsection{Instruction Formats}

Each instruction occupies one word (32 bits) of memory. There are two
instruction formats, A and B, shown in \xf{fmts}. Both formats contain a
6-bit operation code. Format~A contains three register operands and
Format~B contains two register operands and a 16-bit data operand.
Formats are not mentioned further in this document because they are not
relevant to assembly language programming. In general, however, an
instruction is Format~B if and only if it contains a $K$ operand.

\begin{figure}[hbtp]
\begin{center}
\begin{picture}(380,80)
\thicklines

\put( 60,60){\line(1,0){320}}
\put( 60,40){\line(1,0){320}}
\put( 60,40){\line(0,1){20}}
\put(120,40){\line(0,1){20}}
\put(160,40){\line(0,1){20}}
\put(200,40){\line(0,1){20}}
\put(240,40){\line(0,1){20}}
\put(380,40){\line(0,1){20}}
\put( 60,65){\makebox(0,0)[l]{\scriptsize  0}}
\put(120,65){\makebox(0,0)[l]{\scriptsize  6}}
\put(160,65){\makebox(0,0)[l]{\scriptsize 10}}
\put(200,65){\makebox(0,0)[l]{\scriptsize 14}}
\put(240,65){\makebox(0,0)[l]{\scriptsize 18}}
\put(380,65){\makebox(0,0)[l]{\scriptsize 32}}
\put( 30,50){\makebox(0,0){Format A}}
\put( 90,50){\makebox(0,0){opcode}}
\put(140,50){\makebox(0,0){$Ri$}}
\put(180,50){\makebox(0,0){$Rj$}}
\put(220,50){\makebox(0,0){$Rk$}}

\put( 60,20){\line(1,0){320}}
\put( 60, 0){\line(1,0){320}}
\put( 60, 0){\line(0,1){20}}
\put(120, 0){\line(0,1){20}}
\put(160, 0){\line(0,1){20}}
\put(200, 0){\line(0,1){20}}
\put(220, 0){\line(0,1){20}}
\put(380, 0){\line(0,1){20}}
\put( 60,25){\makebox(0,0)[l]{\scriptsize  0}}
\put(120,25){\makebox(0,0)[l]{\scriptsize  6}}
\put(160,25){\makebox(0,0)[l]{\scriptsize 10}}
\put(200,25){\makebox(0,0)[l]{\scriptsize 14}}
\put(220,25){\makebox(0,0)[l]{\scriptsize 16}}
\put(380,25){\makebox(0,0)[l]{\scriptsize 32}}
\put( 30,10){\makebox(0,0){Format B}}
\put( 90,10){\makebox(0,0){opcode}}
\put(140,10){\makebox(0,0){$Ri$}}
\put(180,10){\makebox(0,0){$Rj$}}
\put(300,10){\makebox(0,0){$K$}}

\end{picture}
\end{center}
\caption{Instruction Formats}
\label{fmts}
\end{figure}

Instructions are divided into three classes: data access, arithmetic,
and control. The following subsections describe the effects of each
instruction. Unless otherwise stated, \ic\ is incremented by 4 during
the execution of an instruction. That is, the operation
$\ic\ass{32}\ic+4$ is performed implicitly.

\subsubsection{Data Access Instructions}

See \xf{datac}. The effective address produced by the operand $K(Rj)$ is
$\wreg{j}+K$. The effective address must be legal; otherwise the
processor halts with an error condition. The data field $K$ is
interpreted as a signed, 16-bit quantity: $-16384\le K<16384$.

The effective address of a load word (\kw{lw}) or store word (\kw{sw})
instruction must be aligned; otherwise the processor halts with an error
condition.

A \kw{lb} instruction affects only the 8 low-order bits of the register;
the 24 high-order bits are unaffected.

\begin{figure}[hbtp]
\begin{center}
\begin{tabular}{|l|ll|l|c|} \hline
Function   & \multicolumn{2}{c|}{Operation}           & \multicolumn{1}{c|}{Effect} \\ \hline\hline
Load word  & \kw{lw}  & $Ri,K(Rj)$ & $ \wreg{i} \ass{32} \wmem{\wreg{j}+K} $        \\
Load byte  & \kw{lb}  & $Ri,K(Rj)$ & $ \breg{24..31}{i} \ass{8} \bmem{\wreg{j}+K} $ \\
Store word & \kw{sw}  & $K(Rj),Ri$ & $ \wmem{\wreg{j}+K} \ass{32} \wreg{i} $        \\
Store byte & \kw{sb}  & $K(Rj),Ri$ & $ \bmem{\wreg{j}+K} \ass{8} \breg{24..31}{i} $ \\ \hline
\end{tabular}
\end{center}
\caption{Data Access Instructions}
\label{datac}
\end{figure}

\subsubsection{Arithmetic Instructions}

Most of the arithmetic instructions have three operands. The first two
operands are registers and the third is either a register (\xf{reg}) or
an immediate operand (\xf{imm}) whose value is stored in the
instruction. The first operand receives the result of the operation; the
other operands are not affected by the operation.

\begin{figure}[hbtp]
\begin{center}
\begin{tabular}{|l|ll|l|c|} \hline
Function         & \multicolumn{2}{c|}{Operation}            & \multicolumn{1}{c|}{Effect} \\ \hline\hline
Add              & \kw{add} & $Ri,Rj,Rk$ & $\wreg{i} \ass{32} \wreg{j} +      \wreg{k} $ \\
Subtract         & \kw{sub} & $Ri,Rj,Rk$ & $\wreg{i} \ass{32} \wreg{j} -      \wreg{k} $ \\
Multiply         & \kw{mul} & $Ri,Rj,Rk$ & $\wreg{i} \ass{32} \wreg{j} \times \wreg{k} $ \\
Divide           & \kw{div} & $Ri,Rj,Rk$ & $\wreg{i} \ass{32} \wreg{j} \div   \wreg{k} $ \\
Modulus          & \kw{mod} & $Ri,Rj,Rk$ & $\wreg{i} \ass{32} \wreg{j} \bmod  \wreg{k} $ \\
And              & \kw{and} & $Ri,Rj,Rk$ & $\wreg{i} \ass{32} \wreg{j} \land  \wreg{k} $ \\
Or               & \kw{or } & $Ri,Rj,Rk$ & $\wreg{i} \ass{32} \wreg{j} \lor   \wreg{k} $ \\
Not              & \kw{not} & $Ri, Rj$   & $\wreg{i} \ass{32} \lnot\wreg{j}            $ \\
Equal            & \kw{ceq} & $Ri,Rj,Rk$ & $\wreg{i} \ass{32} \wreg{j} =      \wreg{k} $ \\
Not equal        & \kw{cne} & $Ri,Rj,Rk$ & $\wreg{i} \ass{32} \wreg{j} \ne    \wreg{k} $ \\
Less             & \kw{clt} & $Ri,Rj,Rk$ & $\wreg{i} \ass{32} \wreg{j} <      \wreg{k} $ \\
Less or equal    & \kw{cle} & $Ri,Rj,Rk$ & $\wreg{i} \ass{32} \wreg{j} \le    \wreg{k} $ \\
Greater          & \kw{cgt} & $Ri,Rj,Rk$ & $\wreg{i} \ass{32} \wreg{j} >      \wreg{k} $ \\
Greater or equal & \kw{cge} & $Ri,Rj,Rk$ & $\wreg{i} \ass{32} \wreg{j} \ge    \wreg{k} $ \\
\hline
\end{tabular}
\end{center}
\caption{Arithmetic Instructions with Register Operands}
\label{reg}
\end{figure}

\begin{figure}[hbtp]
\begin{center}
\begin{tabular}{|l|ll|l|c|} \hline
Function         & \multicolumn{2}{c|}{Operation}            & \multicolumn{1}{c|}{Effect} \\ \hline\hline
Add immediate              & \kw{addi} & $Ri,Rj,K$ & $\wreg{i} \ass{32} \wreg{j} +      K $ \\
Subtract immediate         & \kw{subi} & $Ri,Rj,K$ & $\wreg{i} \ass{32} \wreg{j} -      K $ \\
Multiply immediate         & \kw{muli} & $Ri,Rj,K$ & $\wreg{i} \ass{32} \wreg{j} \times K $ \\
Divide immediate           & \kw{divi} & $Ri,Rj,K$ & $\wreg{i} \ass{32} \wreg{j} \div   K $ \\
Modulus immediate          & \kw{modi} & $Ri,Rj,K$ & $\wreg{i} \ass{32} \wreg{j} \bmod  K $ \\
And immediate              & \kw{andi} & $Ri,Rj,K$ & $\wreg{i} \ass{32} \wreg{j} \land  K $ \\
Or immediate               & \kw{ori} &  $Ri,Rj,K$ & $\wreg{i} \ass{32} \wreg{j} \lor   K $ \\
Equal immediate            & \kw{ceqi} & $Ri,Rj,K$ & $\wreg{i} \ass{32} \wreg{j} =      K $ \\
Not equal immediate        & \kw{cnei} & $Ri,Rj,K$ & $\wreg{i} \ass{32} \wreg{j} \ne    K $ \\
Less immediate             & \kw{clti} & $Ri,Rj,K$ & $\wreg{i} \ass{32} \wreg{j} <      K $ \\
Less or equal immediate    & \kw{clei} & $Ri,Rj,K$ & $\wreg{i} \ass{32} \wreg{j} \le    K $ \\
Greater immediate          & \kw{cgti} & $Ri,Rj,K$ & $\wreg{i} \ass{32} \wreg{j} >      K $ \\
Greater or equal immediate & \kw{cgei} & $Ri,Rj,K$ & $\wreg{i} \ass{32} \wreg{j} \ge    K $ \\
Shift left  & \kw{sl} & $Ri,K$ & $\wreg{i} \ass{32} \wreg{i} \ll K$ \\
Shift right & \kw{sr} & $Ri,K$ & $\wreg{i} \ass{32} \wreg{i} \gg K$ \\ \hline
\end{tabular}
\end{center}
\caption{Arithmetic Instructions with an Immediate Operand}
\label{imm}
\end{figure}

The operands need not be distinct. For example, the instruction
\kw{sub}~$R2,R2,R2$ could be used to set register 2 to zero.

The \moon\ processor does not detect carry or overflow in arithmetic
instructions.

The ``logical'' operations, \kw{and}, \kw{or}, and \kw{not}, operate on
each bit of the word, with the usual interpretations.

The comparison instructions (\kw{c\_\_\_}) are similar to the other binary
operators except that the value they store in the result register is
either 1 (if the comparison yields true), or 0 (if the comparison yields
false).

In instructions with immediate operands (\kw{\_\_\_i}), the operand $K$
is a signed, 16-bit quantity. Negative numbers are sign-extended. For
example, the operand $-1$ is interpreted as $-1_{0..31}$, not as 65535
(its 16-bit value).

The shift instructions (\kw{s\_}) are useful if $0\le K\le 31$; their
effect is undefined otherwise. The operators $\ll$ and $\gg$ have the
same effect as \verb"<<" and \verb">>" in C.

\subsubsection{Input and Output Instructions}

See \xf{io}. The instruction \kw{getc} reads one byte from \id{stdin},
the standard input stream. Similarly, \kw{putc} writes to \id{stdout},
the standard output stream.

\begin{figure}[hbtp]
\begin{center}
\begin{tabular}{|l|ll|l|c|} \hline
Function                 & \multicolumn{2}{c|}{Operation}         & \multicolumn{1}{c|}{Effect} \\ \hline\hline
Get character & \kw{getc} & $Ri$ & $\breg{24..31}{i} \ass{8} \mbox{Stdin}  $ \\
Put character & \kw{putc} & $Ri$ & $\mbox{Stdout} \ass{8} \breg{24..31}{i} $ \\ \hline
\end{tabular}
\end{center}
\caption{Input and Output Instructions}
\label{io}
\end{figure}

\subsubsection{Control Instructions}

See \xf{cins}. The target of a branch instruction (that is, the value
assigned to \ic\ if the branch is taken) must be a legal address;
otherwise the processor halts with an error condition.

The jump-and-link instructions are used to call subroutines; they store
the return address in the specified register and then jump to the given
location.

\begin{figure}[hbtp]
\begin{center}
\begin{tabular}{|l|ll|l|c|} \hline
Function                 & \multicolumn{2}{c|}{Operation}         & \multicolumn{1}{c|}{Effect} \\ \hline\hline
Branch if zero           & \kw{bz}  & $Ri,K$  & if $\wreg{i}=0$ then $\ic\ass{16}\ic+K$       \\
Branch if non-zero       & \kw{bnz} & $Ri,K$  & if $\wreg{i}\ne 0$ then $\ic\ass{16}\ic+K$    \\
Jump                     & \kw{j}   & $K$     & $\ic\ass{16}\ic+K$                            \\
Jump (register)          & \kw{jr}  & $Ri$    & $\ic\ass{32}\wreg{i}$                         \\
Jump and link            & \kw{jl}  & $Ri,K$  & $\wreg{i}\ass{32}\ic+4; \ic\ass{16}\ic+K $    \\
Jump and link (register) & \kw{jlr} & $Ri,Rj$ & $\wreg{i}\ass{32}\ic+4; \ic\ass{16}\wreg{j} $ \\
No-op                    & \kw{nop} &         & Do nothing                                    \\
Halt                     & \kw{hlt} &         & Halt the processor                            \\ \hline
\end{tabular}
\end{center}
\caption{Control Instructions}
\label{cins}
\end{figure}

\subsection{Timing}

The time required to run a program is measured in clock cycles and
dominated by memory access. There are two paths to the memory; one is
used to read instructions and the other is used to read and write data.

Before each instruction is executed, the processor must load a 32-bit
word containing the instruction. This requires 10 clock cycles.

For data, the processor uses a {\it memory address register\/} (MAR) and
a 32-bit {\it memory data register\/} (MDR). The processor loads an
address into MAR and issues a read or write directive to the memory
controller. The memory controller either obtains a word of data from the
memory and stores it in MDR (read) or copies the contents of MDR to the
memory.

A read or write operation requires 10 clock cycles. If the data required
for a read operation is already in the MDR, the read operation requires
only 1 clock cycle. For example, loading the four bytes of a word using
\kw{lb} instructions requires 10 clock cycles for the first byte and
1~clock cycle for each of the other three bytes, provided that no other
data access intervenes.

%%%%%%%%%%%%%%%%%%%%%%%%%%%%%%%%%%%%%%%%%%%%%%%%%%%%%%%%%%%%%%%%%%%%%%%%%%%%%%%%

\section{\moon\ Assembly Language} \label{sec:asslan}

Programs for the \moon\ processor are written in its {\it assembly
language\/}. We use the following typographical conventions to describe
the grammar of the assembly language. \xf{gram} shows the grammar.

\begin{itemize}

\item Non-terminal symbols are written in \nont{slanted type} and have
      an initial upper case letter.
      \linebreak Examples: \nont{Program}, \nont{Instr}.

\item Terminal symbols are written in a \kw{sans serif} font.
      Punctuation symbols are quoted.
      \linebreak Examples: \kw{eol}, ``\verb","''.

\item The following symbols are metasymbols of the grammar:

      \begin{tabular}{cl}
      $\longrightarrow$  & separates the defined symbol from the
                           defining expression; \\
      $|$                & indicates alternatives; \\
      $[ \,\ldots\, ]$   & enclose an optional item (zero or one
                           occurrences); \\
      $\{ \,\ldots\, \}$ & enclose a repeated item (zero or more
                           occurrences). \\
      \end{tabular}

\end{itemize}

\begin{figure}[tbhp]
\begin{center}
\begin{tabular}{|rcl|} \hline
\nont{Program}  & $\longrightarrow$ & \{ \nont{Line} \} \kw{eof} \\
\nont{Line}     & $\longrightarrow$ & [ \nont{Symbol} ] [ \nont{Instr} $|$ \nont{Directive} ] [ \nont{Comment} ] \kw{eol} \\
\nont{Directive}& $\longrightarrow$ & \nont{DirCode} [ \nont{Operand} \{ ``\verb","'' \nont{Operand} \} ] \\
\nont{Instr}    & $\longrightarrow$ & \nont{Opcode} [ \nont{Operand} \{ ``\verb","'' \nont{Operand} \} ] \\
\nont{Operand}  & $\longrightarrow$ & \nont{Register} $|$ \nont{Constant} [ ``\verb"("'' \nont{Register} ``\verb")"'' ] $|$ \nont{String} \\
\nont{Register} & $\longrightarrow$ & ( ``\verb"r"'' $|$ ``\verb"R"'' ) \nont{Digit} [ \nont{Digit} ] \\
\nont{Constant} & $\longrightarrow$ & \nont{Number} $|$ \nont{Symbol} \\
\nont{Number}   & $\longrightarrow$ & [ ``\verb"+"'' $|$ ``\verb"-"'' ] \nont{Digit} \{ \nont{Digit} \} \\
\nont{String}   & $\longrightarrow$ & ``\verb+"+'' \{ \nont{Char} \} ``\verb+"+'' \\
\nont{Symbol}   & $\longrightarrow$ & \nont{Letter} \{ \nont{Letter} $|$ \nont{Digit} \} \\
\hline
\end{tabular}
\end{center}
\caption{Assembly Language Grammar}
\label{gram}
\end{figure}

The symbols \kw{eof} and \kw{eol} denote ``end of file'' and ``end of
line'', respectively.

A \nont{Symbol} is a string consisting of the following characters:
letters, digits, and \underline{\hskip 0.7em}. The first character of a
symbol must not be a digit. Directives and instruction codes must not be
used as symbols. Strings of the form ``\verb"R"$\,\{\,\id{Digit}\,\}$''
and ``\verb"r"$\,\{\,\id{Digit}\,\}$'' are not legal symbols
(cf.~register syntax below).

A \nont{Comment} starts with the character ``\cmt'' and continues to the
end of the current line.

A \nont{Constant} is a signed, decimal number.

The registers are ``\verb"R0"'' through ``\verb"R15"''. The letter ``R''
may be either upper or lower case.

The predefined symbol \id{topaddr} has $M+1$ as its value, where $M$ is
the highest legal address. This symbol can be used to check for
addressing errors or to initialize a stack or frame pointer. For
example, the following instruction could be used to initialize the frame
pointer:

\begin{prog}
\tt\-\-addi\-r14,r0,topaddr
\end{prog}

The syntax of \nont{Directive} depends on the particular directive, as
shown in \xf{dir}.

\begin{figure}[hbtp]
\begin{center}
\begin{tabular}{|ll|l|} \hline
\multicolumn{2}{|c|}{Directive} & \multicolumn{1}{|c|}{Effect} \\ \hline\hline
\kw{entry} &                  & The following instruction will be the first to execute \\
\kw{align} &                  & The next address will be aligned \\
\kw{org}   & $K$              & The next address will be $K$ \\
\kw{dw}    & $K_1,K_2,\ldots$ & Store words in memory \\
\kw{db}    & $K_1,K_2,\ldots$ & Store bytes in memory \\
\kw{res}   & $K$              & Reserve $K$ bytes of memory \\ \hline
\end{tabular}
\end{center}
\caption{Directives}
\label{dir}
\end{figure}

The operands of a \kw{dw} directive are either symbols or integers.

The operands of a \kw{db} directive are bytes (unsigned numbers in the
range $0,1,\ldots,255$) or strings enclosed in quotes ({\tt "~...~"}).
The characters in the string must be ASCII graphic characters (codes 32
through 126) only. The \moon\ simulator does not recognize escape
characters in strings.

The operand of a \kw{res} directive is a positive integer, $K$. The
assembler requires $K<2^{31}$, but in practice the maximum value of $K$
will be limited by the amount of memory available.

\xf{prog} shows a listing that might be generated by the assembler for a
simple program. The addresses in the left column would not be included
in the input file generated by a programmer or compiler.

\begin{figure}[hbtp]
\begin{verbatim}
          1     0         org   103
          2   103 message db    "Hello, world!", 13, 10, 0
          3   119         org   217
          4   217         align
          5   220         entry                  % Start here
          6   220         add   r2,r0,r0
          7   224 pri     lb    r3,message(r2)   % Get next char
          8   228         ceqi  r4,r3,0
          9   232         bnz    r4,pr2          % Finished if zero
         10   236         putc   r3
         11   240         addi   r2,r2,1
         12   244         j      pri             % Go for next char
         13   248 pr2     addi   r2,r0,name      % Go and get reply
         14   252         jl     r15,getname
         15   256         hlt                    % All done!
         16   260
         17   260 % Subroutine to read a string
         18   260 name    res    59              % Name buffer
         19   319         align
         20   320 getname getc   r3              % Read from keyboard
         21   324         ceqi   r4,r3,10
         22   328         bnz    r4,endget       % Finished if CR
         23   332         sb     0(r2),r3        % Store char in buffer
         24   336         addi   r2,r2,1
         25   340         j      getname
         26   344 endget  sb     0(r2),r0        % Store terminator
         27   348         jr     r15             % Return
         28   352
         29   352 data    dw     1000, -35
         30   360         dw     99, getname
\end{verbatim}
\caption{An Assembly Language Program}
\label{prog}
\end{figure}

The program begins with a directive, \kw{org}, specifying that the data
labelled ``{\tt message}'' will be stored at address 103. Since the
message is a byte string, it is in fact stored at that address, without
alignment.

The processor and assembly language do not require any particular format
for strings. The convention used in this program is that strings are
null-terminated, as in C. An alternative would be to prefix a string
with a number giving its length, as Pascal does. The bytes 13 and 10 are
{\sc return} and {\sc linefeed}, respectively.

The directive \kw{org~217} sets the current address to 217. The
\kw{align} directive changes the current address to the next word
boundary, 220. The directive \kw{entry} immediately before this
instruction indicates that it is the first instruction to be executed.

The directive \kw{res~59} at address 260 reserves 59~bytes of memory.
The following directive, \kw{align}, ensures that the next instruction
will be aligned on a word boundary.

\section{The \moon\ Simulator}

The assembler/simulator is a program that assembles a \moon\ program and
simulates its execution. It is implemented as a C program (\kw{moon.c}), 
which can be compiled using any standard C compiler, such as \kw{gcc}. 
In this documentation, it is assumed that the name of the program is \kw{moon}. 
In more detail, \kw{moon} performs the following actions.

\begin{itemize}

\item Read the assembly language files indicated on the command line and
      store them in the simulated memory. There will typically be two
      files, a program with subroutines and a subroutine library. The
      loader checks for syntax errors; if any are found, \kw{moon}
      reports the errors and returns without further processing.

\item By default, start executing the program at the entry point and
      continue simulating until a \kw{hlt} instruction has been
      executed. If the user selects the trace option, the simulator
      enters trace mode.

\end{itemize}

The simulator is invoked by a command of the form

\begin{prog}
\kw{moon} $a_1,\ldots,a_n$
\end{prog}

in which $a_1,a_2,\ldots$ are command-line arguments. The arguments may
appear in any order; \xf{args} describes the permitted values and effect
of each argument. The default values of arguments are indicated by
bullets between the argument and its description.

The \kw{p} directive may be used to generate listings of selected files.
For example, the command

\begin{prog}
\kw{moon} $+\kw{p}$ \kw{main} $-\kw{p}$ \kw{lib}
\end{prog}

would generate a listing of \kw{main} but not of \kw{lib}. The listing
will be written to \verb"moon.prn" unless a file name is provided with a
$+\kw{o}$ argument. There must not be any blanks between the \kw{o}
directive and the file name.

\begin{figure}[hbtp]
\begin{tabular}{lcp{5in}}
$\ang{filename}$        && Read assembly language from the file
                           $\ang{filename}$, assemble it, and store
                           it in memory. If the filename has no
                           extension, \moon\ adds \kw{.m}. \\
$+\kw{p}$               && Generate a listing. \\
$-\kw{p}$               &$\bullet$
                         & Do not generate a listing. \\
$+\kw{s}$               && Display values of symbols. \\
$-\kw{s}$               &$\bullet$
                         & Do not display values of symbols. \\
$-\kw{t}$               &$\bullet$
                         & Execute the program in normal mode. \\
$+\kw{t}$               && Execute the program in trace mode. \\
$-\kw{x}$               && Do not execute the program. \\
$+\kw{x}$               &$\bullet$
                         & Execute the program. \\
$+\kw{o}\ang{filename}$ && Write listings to $\ang{filename}\kw{.prn}$. \\
\end{tabular}
\caption{Command-line Arguments}
\label{args}
\end{figure}

If \kw{moon} is started in trace mode, it responds interactively to the
commands described in \xf{trac}. Command letters may be entered in upper
or lower case. The operand of a trace command, shown as $\ang{m}$, may
be given as a number or a symbol. For example, if we were tracing the
program of \xf{prog}, either of the commands

\begin{prog}
\kw{b}320
\kw{b}getname
\end{prog}

would set a breakpoint at address 320. The case of letters in symbol
names is significant.

\begin{figure}[hbtp]
\begin{tabular}{lp{5in}}
{\sc return}           & Execute $k$ instructions, where $k$ is 10 by
                         default but can be changed by the \kw{k} command. \\
$\kw{b}$               & Show all breakpoints. \\
$\kw{b}\ang{m}$        & Set a breakpoint at memory location $m$. \\
$\kw{c}$               & Clear all breakpoints. \\
$\kw{c}\ang{m}$        & Clear the breakpoint at memory location $m$. \\
$\kw{d}$               & Dump memory locations $\ic\pm20$. \\
$\kw{d}\ang{m}$        & Dump memory locations $m\pm20$. \\
$\kw{i}$               & Set \ic\ to entry point. \\
$\kw{i}\ang{m}$        & Set \ic\ to $m$. \\
$\kw{h}$               & Display a help screen. \\
$\kw{k}$               & Set $k$ to its default value of 10. \\
$\kw{k}\ang{m}$        & Set $k$ to $m$. \\
$\kw{q}$               & Quit the simulator. \\
$\kw{r}$               & Display register values. \\
$\kw{s}$               & Display symbol values. \\
$\kw{x}$               & Run until next breakpoint. \\
$\kw{x}\ang{m}$        & Run until $\ic=m$. \\
\end{tabular}
\caption{Interactive Commands for Trace Mode}
\label{trac}
\end{figure}

In trace mode, \kw{moon} initializes the value of \ic\ to the entry point
and maintains it in accordance with instructions executed thereafter. As
each instruction is executed, the interpreter displays the instruction
and the values of changed registers or memory locations. The command
\kw{i} sets the value of \ic\ to the given value, or to the entry point
if no value is given.

The command \kw{d} displays values of memory words and the command
\kw{r} displays values of registers. Each value is displayed as a
hexadecimal number, as a string of four characters, and as a 32-bit
signed integer. In the character display, non-graphic characters are
shown as dots.

\subsection{Programming Conventions}

The \moon\ architecture does not restrict programmers to any particular
pattern of use. The addressing mode $K(Ri)$ is suitable for addressing a
stack, with $Ri$ as the stack pointer and $K$ as an offset computed by
the compiler. Any register can be used as the link for subroutine calls.
Arguments can be passed either in registers or on the stack.

\subsection{Defects of the Simulator}

The precise behaviour of the \moon\ simulator depends on the
architecture of the processor on which it is running and also on the C
compiler used to compile it.

\begin{itemize}

\item The order in which bytes are stored in a word is inherited from
      the host processor. This does not affect the execution of \moon\
      instructions but does affect the order in which characters are
      displayed during tracing.

\item The shift instructions (\kw{sl} and \kw{sr}) of the \moon\
      processor are simulated using the C operators \verb"<<" and
      \verb">>". The effect of \verb">>" is undefined when the most
      significant bit of the left operand is set. Right-shifting a
      negative number may yield either a positive or a negative number.

\item The effect of the \kw{putc} and \kw{getc} instructions depends on
      whether the simulator is running in normal or trace mode. In
      normal mode, \kw{getc} reads a string from the keyboard and yields
      one character of the string each time it is executed. In trace
      mode, you should enter characters one at a time, as \kw{getc} asks
      for them.

\end{itemize}
\end{document}
